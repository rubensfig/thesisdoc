\section{Software Defined Networking} \label{chap:sdn} %% chapter 3

Computer networking is a vital part of the services that are offered today, and as such, the performance in technology backing these services is central to the quality of these services. As the service providers move their
data centers to cloud computing environments, enabling several improvements in the predictability, quality of service and ease of use of their services, new technologies are required to make sure that their services are adapted
to the fast changing landscape of networking services. One of the most notable innovations in this field is called \textbf{Software Defined Networking}, which, as described by the Open Networking Foundation, \textit{ In the SDN 
architecture, the control and data planes are decoupled, network intelligence and state are logically centralized, and the underlying network infrastructure is abstracted from the applications} \cite{CITE this - onf website}. 
The two main contributions of this architecture is:

\begin {itemize}
    \item \textbf{Separation of network planes} SDN allows for the separation of the network control plane from the data forwarding plane by having network "intelligence" present in the network controllers, and having them
control the forwarding elements that live in the Data Plane
    \item \textbf{Centralization of network management functions} By isolating the management on a separate plane, there is possibility of developing a single controller that can regulate the entire network, having unrestricted access 
        to every element present in the network, simplifying management, monitoring, application of QoS policies, flow optimization, ...
\end {itemize}

\par This new paradigm introduces programmability in the configuration and management of networks, by consolidating the control of network devices to a single central controller, achieving separation of the control and the 
data plane, and supporting a more dynamic and flexible infrastructure. Another important development following the development of SDN is the concept of Network Function Virtualization. This concept allows to remove 
\textit {middleboxes}\footnote {Computer networking device that does some operations on traffic, excepting packet forwarding. Examples include caches, IDS's, NAT's, ..}, by replacing these with generic software applications.

\begin{figure}[!tbph]
  \centering
  \subfloat[Traditional networking architecture]{\includegraphics[width=0.4\textwidth]{bib/network_trad}\label{fig:net_trad}}
  \hfill
  \subfloat[SDN architecture]{\includegraphics[width=.4\textwidth]{bib/network_sdn}\label{fig:net_sdn}}
  \caption {Traditional vs SDN network architecture}
\end{figure}

\par By moving network infrastructure to SDN models, the difficulty of managing a network is greatly reduced, since the logical centralization of the control layer exposes the global view of the network, simplifying management tasks. 
Furthermore, this also removes the challenge of configure each networking device individually, turning network operation and management into setting high level policies in the controllers, and letting 
the protocols that handle connection between the devices and controllers set the actual rules. 

\par Software-Defined Networking is defined as being composed of two layers: \textbf{Northbound Interfaces}, which are composed of Application Programming Interfaces (API) for communication between applications and the controller,
enabling network services like routing, security, visualization and management; and the \textbf {Southbound Interfaces} which main role is to connect the network devices between the controllers and network via protocols like
OpenFlow (see section \ref{sec:of}), or P4 \footnote {https://p4.org/}. 

\begin{figure}[!tbph]
  \centering
  \includegraphics[width=0.4\textwidth]{sdn/sdn_division}
  \label{fig:sdn_div}}
  \caption {SDN Interfaces division}
\end{figure}

\par Understanding SDN platforms is then composed of understanding the operation of both interfaces, and defining requirements for their operation, which are listed below \cite {CITE - sdn_reference_architecture_open_api.pdf}.
These requirements are general principles for networks, but the addition of the SDN controller introduces another point of failure, that could be damaging to the entire network.

\begin {enumerate} 
    \itemsep0em
    \item High Performance   
    \item High Availability 
    \item Fault Tolerance   
    \item Monitoring   
    \item Programmability   
    \item Modularity   
    \item Security   
\end {itemize}

\subsection {OpenFlow} \label{sec:of}

With the growth of the networking infrastructure of the past few decades, the need for an environment that allows for experimentation and testing of different protocols and equipment became evident. If networking 
research would depend on the previously existing methods, then new ways of creating and developing protocols would become increasingly hard to implement and develop. As such, there was need for a framework that could 
enable testing of new ideas on close to realistic settings. So, on February 2011 the version 1.1 of OpenFlow was released, and this proposal quickly became the standard for networking in a Software Defined Network. Since
2011, this protocol has suffered some revisions, and the latest version supported is version 1.5.1. 

\par Several reasons led to the quick standardization of this protocol, which are related not only to the initial requirements of the platform, like the capability of supporting high-performance and low-cost implementations, but 
also the extensibility that the open source development model provides, removing the limitations that closed or commercial solutions give the network researchers.

\par The big advantage of OpenFlow is that it is, from the data forwarding point of view, easy to process. Since the control decisions are made by the controller, which lives in a separate plane, all the switch needs to do
is correctly match the incoming packets, and forward them according to the rules established by the controller. The components that are part of this system and enable this functionality are:

\begin {itemize}
    \item \textbf {FlowTables} This element describes the main component of the switching capabilities of the OpenFlow switch. Inside the switch there are several flow tables that can be used to match incoming packets,
and process them in the rules that are specified by the controller. These rules can contain actions that affect the path of the packets, and these actions usually include forwarding to a port, packet modification, among
others. Classification is done via matching one or more field present in the packet, for example the switch input port, the MAC and IP addresses, IP protocol, basically all information required to correctly process the 
incoming packet. The required actions for an OpenFlow switch are the capability of forwarding to a set of output ports, allowing the packet to move across the network; to send them to the controller, in the case of a
miss of match; and finally the ability to drop packets, which is useful for DDoS mitigation, or more security concerns.
    \item \textbf {OpenFlow Protocol} Through the establishment of the OpenFlow Protocol between the switch and the controller, there is the definintion of several messages that allow for the control of the switch. This protocol
enables capabilities such as adding, deleting and updating flow mods in the switch, that are referred to as \textit {Controller-to-Switch} messages. Other relevant message types are the \textit {Asynchronous}, that enable the
notification of some event that occurred, this type includes the Packet-In message, that is a type of message that is sent to the controller when a certain packet has no match in the flow tables present in the switch; and the 
\textit{ Synchronous} message that enable functionality such as the Hello message, that is used to start the connection between the switch and the controller.
    \item \textbf {Secure Channel} OpenFlow defines the channel that is between the switch and the controller as a secure communications channel. As the messages that are sent to the switch are critical for the correct operation 
of the system, as indicated in the previous point, the channel should be cryptographically secure, to prevent spoofing of this information. As such, the channel is tipically transported over TLS.
\end {itemize}

% XXX SOURCE This images were taken from : open flow switch specification  https://3vf60mmveq1g8vzn48q2o71a-wpengine.netdna-ssl.com/wp-content/uploads/2014/10/openflow-switch-v1.3.5.pdf
\begin{figure} [h]
    \centering
    \begin{subfigure}
    \includegraphics[width=0.25\textwidth]{sdn/open_flow_switch_pipeline}
    \end{subfigure}
    \begin{subfigure}
    \includegraphics[width=0.6\textwidth]{sdn/open_flow_tables}
    \end{subfigure}
    \caption{Images describing OpenFlow components. On the left, an overview to the entire system, and on the right a view at the table structure of the OpenFlow Switch}
\end{figure}

\subsection {Network Devices}

Networking devices are a central part of network operation, performing routing, switching, management operations, and span the different layers of the OSI model. The investment in dedicated hardware to perform management functions can
potentially be replaced by SDN controllers, offloading the QoS policies and traffic engineering (TE) functions from hardware to software. As such, in this section we focus on the devices that are responsible for the operation on 
layers 2 and 3 of the standardized networking stack, switches and routers. A networking switch is a device that connects multiple devices on a computer network, using hardware addresses (MAC) to forward data inside the network, 
by mapping each port with a certain MAC address, while a router is responsible of forwarding packets between different computer networks. These devices run at different layers of the networking stack, the former operating at the 
data-link, or layer 2, and the latter operating at layer 3, or network. This is not a clear separation however with multilayer devices, where switches also provide routing capabilities. Typical vendors for these solutions include
Cisco, Juniper, but the rise of whitebox switches that have support for deployments in SDN environments enables network operators to avoid vendor lock-in, and take advantage of the open nature of these devices. Commonly associated 
with whitebox switches is the support for the OpenFlow protocol, making these an essential part of the SDN infrastructures.

\par The performance \footnote {Defined as the throughput and latency of the network node} of networking devices is central to the proper operation of networks, especially in deployments in Data Centers, where the 
interfaces must be able to support 100 Gbps links and further, while also maintaining the programmability that is expected of SDN based infrastructures. This performance is linked with the hardware that is chosen to serve as the 
base for the devices \cite {CITE - sdn_challenges}:

\begin {enumerate} 
    \itemsep0em
    \item \textbf{General Purpose Processors} provide the greatest flexibility of all the solutions, while providing the worst results in performance, due to the general purpose design of the hardware and the optimizations present
        in the other architectures.
    \item \textbf{Field-Programmable gate arrays (FPGA)} are a platform that enable the configuration of the devices via hardware design tools, maintaining the programmability of the GPPs, while also allowing for designing the
        devices around the tasks that they perform, including optimizations for switching/ routing. A notable example of platforms based in these systems is NetFPGA \footnote {https://netfpga.org/site/\#/}, an open source 
        hardware platform designed for research, and supporting up to 100G operation. 
    \item \textbf{Application-specific integrated circuits (ASIC)} are integrated circuits that are customized for one particular application, removing the programmability, but also providing the greatest performance of the former 
        options.
\end {itemize}

\par These architectures generally allow to design SDN architectures around general purpose hardware, contributing to the flexibility of this paradigm, even considering the proprietary nature of the ASICs, which can be bundled with 
SDKs for developing other applications on top of these. 

\par Considering OF enabled hardware switches, the processing of incoming packets is done as by matching a (up to) 15 field tuple \cite {CITE - flow_table_management_sdn_switches} to several flow tables, 
that have rules sent from the controller. In these cases, the possibility of bottlenecks are due to several factors, including the latency of the installation of new flow rules, and the memory limitation on the hardware. Solutions 
to the memory limitations in OF switches include DevoFlow \cite {CITE - https://hal.inria.fr/hal-00825087/document}, which utilizes wildcard rules to reduce the number of flow entries that are installed on the devices, while also
aggregating traffic, which simplifies detection and management of the elephant flows, due to reduced control plane load; and SmartTime \cite {CITE - http://rishabhpoddar.com/publications/SmartTime.pdf} manages the timeouts
for the rules on the switch, reducing this in the presence of microflows, and increasing the timeout in the case of the occurrence of longer lived flows, which improves memory utilization and reduces the load on the controller.

\par Virtualised environments also allowed the development of Software Switches, due to the highly dynamic nature of virtual environments, where Virtual Machines (VMs) can move between physical compute nodes and frequent
network topology changes. Furthermore, standard Linux bridges cannot handle the multi-server deployments \footnote {https://github.com/openvswitch/ovs/blob/master/Documentation/intro/why-ovs.rst} used in virtualised environments. 
Open vSwitch is a soft switch that can join traditional switches on these platforms, replacing them where these couldn't be deployed. OVS switches are also compliant with the OpenFlow protocol, which clearly shows the flexibility
that can be achieved by combining all the networking devices with one management protocol. 

\subsection {SDN Controllers}

Central for operation of the networks, SDN controllers allow the orchestration of the multiple parts required for correctly operating a large scale network. Separate from the data traffic on the network, these are responsible for the interaction
between the Northbound networking applications and the Southbound devices, as described in figure \ref{fig:sdn_controller_arch}. 

\begin{figure}[!tbph]
  \centering
  \includegraphics[width=0.4\textwidth]{sdn/sdn_controller_arch}
  \label{fig:sdn_controller_arch}
  \caption {SDN Interfaces division}
\end{figure}

\par Although the use case of the controllers will depend on each deployment and implementation, the basic use case is to provide connectivity across layer 2 and 3 networks, which is achieved via flow management, including operations like
switching, forwarding and potentially load balancing. The logically centralized position augments this capability by keeping the state of the entire network, which facilitates route planning and management. 

\par Despite the advantages that the controllers centralization provides, this also introduces a Single Point of Failure (SPOF) \cite{ CITE - https://arxiv.org/pdf/1308.6138.pdf}, exposing a weakness to Denial-of-Service (DoS) attacks and 
controller failure. The potential catastrophic scenario related to controller downtime due to these failures means that an approach must be planned for disaster failure and recovery. High Availability setups are used to mitigate the potential
failure of the controllers, by having multiple backups running. In order to ensure that the same network state in every controller, every switch must be connected to every controller, but as specified in section \ref{sec:fault_tolerance}, only the Master 
controller writes the messages to the networking devices, which ensures that duplicate rules are not enforced. In case of controller failure, one of the backup controllers can take over the role of the previous master, without any outages on the network.

% XXX - CONTROLLER RESEARCH

\subsubsection{Existing Platforms}

There are several controller implementations available for use, each with different interfaces, performance, and modularity. In \cite{CITE - https://pdfs.semanticscholar.org/1fae/98c9d3bed22a539155d5e93a8d420bd22837.pdf} a comparative study is
performed on available SDN controllers at the time, and compare the different characteristics of each controller, like the available interfaces, the language of implementation, modularity, etc. In the following sections, we explore two of the 
highest rated solutions, Floodlight and OpenDaylight.

\subsubsection {Floodlight}

\begin{figure}[!tbph]
  \centering
  \subfloat[Floodlight GUI]{\includegraphics[width=0.5\textwidth]{sdn/floodlight_gui}\label{fig:flood_gui}}
  \hfill
  \subfloat[Floodlight Topology]{\includegraphics[width=.5\textwidth]{sdn/floodlight_topology}\label{fig:flood_topo}}
  \caption {Floodlight Web Views}
\end{figure}

Floodlight \footnote{http://www.projectfloodlight.org/floodlight/} is a java-based SDN controller, and is one the first open-source solutions to gain relevance in research and industry 
\cite{http://zoo.cs.yale.edu/classes/cs434/cs434-2017-spring/readings/onos-hotsdn2014.pdf}. This controller provides the OpenFlow interface, and it enables adding several modules, either through extensions, or through the utilization of the provided
REST interface, simplifying the addition of new features to the base controller.

\par The REST interface provides an API to interact with the switch, which allows developers to get statistics, push flow entries, and more. It also provides an useful GUI for easier visualization of the topology, link state, and port statistics.
 
\subsubsection {OpenDaylight} \label{chap:odl}

OpenDaylight \footnote{https://www.opendaylight.org/} (ODL) is a major project supported by the major vendors. The main differences between ODL and other controllers is support for other protocols in the southbound interface, 
due to the creation of a Service Abstraction Layer (SAL) \cite{CITE - opendaylight_sdn_controller_architecture}. In a high level overview, the creation of the Model-Driven SAL allows to extend the controller basic functionality
by the addition of several plugins, which are used in combination with a RESTCONF interface, defining the data models for the data stores, and the RPCs for interaction between the data.

\begin{figure}[!tbph]
  \centering
  \includegraphics[width=.8\textwidth]{sdn/odl_topology}
  \label{fig:odl_topo}
  \caption {OpenDaylight Topology}
\end{figure}
