\chapter{Software Defined Networking} \label{chap:sdn} %% chapter 3

Computer networking is a vital part of the services that are offered today, and as such, the performance in technology backing these services is central to the quality of these services. As the service providers reorganize their
data centers in the cloud computing domain, enabling several improvements in the predictability, quality of service and ease of use of their services. New technologies are then required to make sure that their services are adapted
to the fast changing landscape of networking services. One of the most notable innovations in this field is called Software Defined Networking, because its architecture allows for two essential features

\begin {itemize}
    \item \textbf{Separation of network planes} SDN allows for the separation of the network control plane from the data forwarding plane by having network "intelligence" present in the network controllers, and having them
control the forwarding elements that live in the Data Plane
    \item \textvf{Centralization of network management functions} By isolating the management on a separate plane, there is possibility of developing a single controller that can regulate the entire network, having unrestricted access to every element present in the network, simplifying management, monitoring, application of QoS policies, flow optimization, ...
\end {itemize}

In this chapter we explore the essential characteristics of SDN, the technologies that provide the back end for the development of this technology, and current implementations of the most popular SDN controllers, so that we can 
see the features that should be present while developing a management interface for SDN controllers.

\section {Switches}
\subsection {OVS}
\section {OpenStack}
\section {OpenFlow}
\section {SDN Controllers}
\subsection {Floodlight}
\subsection {OpenDaylight}
\section {SDN Northbound}
