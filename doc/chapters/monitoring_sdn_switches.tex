\chapter{Monitoring SDN Switches} \label{chap:mon_sdn} %% chapter 3

\section {Problem}

Developing solutions for use in mission critical environments requires the deep understanding and analysis of the requirements of these environments, like those
present in large data centers. The Basebox system (see section \ref{sec:bisdn}) is an example of these solutions, but lacks a system for monitoring and management 
of the network devices. Such a system should 

\begin {itemize}
    \item display the topology information reported by CAWR, including the internal switch links, and the LACP discovered bond interfaces on the servers,
    \item display the port and link statistics for both switches,
    \item design an alerting system, so that network operators can be informed of changes on the network state,
    \item provide some diagnostic capabilities.
\end {itemize}

\par Also addressed by this system should be the definition of Quality-of-Service (QoS) policies, that maintains the levels of accepted traffic behaviour of each 
device in the network, and identifies and applies some automatic mitigation strategy when the system does not perform according to normal state. By understanding
data center traffic characteristics, one of the largest problems are the existence of \textit{elephant flows}, that impact the available bandwidth of the network.
As such, the development of a full management system should also include a system that receives the incoming port statistics, analyses these, and applies
statistical analysis to manage the impact caused by elephant flows.

\section {Solution}

\begin{figure} [h]
    \centering
    \includegraphics[width=0.5\textwidth]{proposed_work/proposed_system}
    \caption{High-level visual description of the proposed system} \label{fig:pro_sys}
\end{figure}

\par Following the previously presented requirements, the proposed system is present in figure \ref{fig:pro_sys}. By adding an interface for the controllers that 
exposes management information like the network's topology, statistics, etc, we are able to build a \textbf{Operations Support System} (OSS) that provides the basis 
for the development of applications that monitor the state of the network. This system allows further separation of roles in the network, in contrast to a system 
where the controller would gather the roles of managing and monitor the network's status. Furthermore, this architecture increases the modularity of the components, 
enabling hot-swapping different modules, and allows parallel development of different features in the monitoring and management stack.

\par We provide a proof-of-concept composed of two components: the first is a Graphical User Interface (GUI) providing an user friendly interface to display 
topology and statistics, and the second is an intelligent system enabling the detection of elephant flows. In this section we describe in a high level way the 
approaches for the development of this system.

\subsection {GUI}

The primary use case for this component is following the changes in the underlying topology, while also allowing the monitoring some aspects of the port statistics,
and as such, the links between switches and the hosts, and the association between the ports and the switches should be displayed. Performance wise, this system must
run as fast as possible, and the transmission of data must not interfere with the systems operation. 

\par In order to reduce the memory requirements of the platform, and decrease the time that it takes for drawing topology updates, a decision was made to not store 
state, which would increase the time it takes for topology updates, with queries to store and load data from these databases. This also removes complexity as the 
system grows, where storing information about links and switches would dramatically increase database size.

\par Motivated by compatibility with standardized systems, choosing the data model for representing the underlying system required the investigation of similar 
systems, and the RFCs, or similar standardized documents that exist. 

\subsection {Operations Support System}

The second component of our system was developed for monitoring the system statistics, and displaying alerts when elephant flows are detected. For this end, we have
implemented an algorithm for the detection part. 

\par The developed system aims to detect the elephant flows before the effect of these is detected in the rest of the network. As we assume the tree topology in data
centers, seen in figure \ref{fig:fattree}, the testing environment must be designed to monitor the lower layer edge switches, connected directly to the physical 
hosts. By monitoring the ports on these switches, we detect changes in the reported traffic statistics via a developed Python script.
