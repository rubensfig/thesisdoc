\chapter{Introduction} \label{chap:introduction} 

\section {Context} \label{context}

The rapid expansion of the cloud computing environment in the previous decade is related to the increasing demand in computational power that applications like 
distributed databases or data analysis have. Public cloud solutions like Amazon's Web Services, Microsoft's Azure, or private solutions offered through OpenStack
provide a very large pool of resources for developers to deploy applications with ease. Through economies of scale, these vendors have
centralized their solutions in very large scale data centers, consolidating their operations in a single location, allowing increased performance of applications, and
easier maintenance. The offer of virtualisation solutions also contributes to the recent surge in popularity of these systems, due to the less spending in
operational costs and improved utilization of hardware \cite{sims_david_carousels_2011}. Subscription based systems are typically available for renting, allowing 
users to use services on a Virtual Machine.

\par Using open source applications and whitebox hardware has also contributed to the success of these environments, due to the possible cost reductions. Software
Defined Networking (SDN), has proven to be a reliable environment to manage data center environments, due to the centralization of the network
controllers, improved programmability of the network's data plane, and improved management systems. Network programmability, despite not being exclusive to the 
SDN framework, eliminates the effort in individually configuring every network device, which in large scale environments becomes an impossible task. This model
also provides the network engineers and software developers a high level network abstraction that is used to monitor network utilization and optimize resource
utilization. 

\par Network management systems (NMSs) provide the central architectural component that allows the system administrators to track the systems utilization, 
analyse link, node and device failures, and observe alarms when the networks' state is outside the range that network operators define as normal. These systems
also provide an insight for network operators to research where resources should be allocated. However, network planning considering virtualisation solutions 
are considered a difficult solution, due to complex scheduling policies and performance deviations \cite{sampaio_energy-efficient_2015}. 

\par The topic of traffic engineering in SDN is central to this field, more specifically, the flow and application level monitoring.  The possibility of acting
differently towards network traffic is an advantage for network operators, allowing to prioritize applications that are
sensitive to latency, like video and audio streaming. As such, focus on service prioritization and delivery optimization has been the focus of research 
of the past few years \cite{bakhshi_user-centric_2017}.

\section {Motivation}

% Outline current methods
% Evaluate current methods

The increased relevance that cloud computing solutions have on today's networking environments, and the growing opportunities in data centers, mainly
kick-started by the prevalence of OpenStack, have brought a rising demand of open sourced solutions that provide an interface in these environments. Exposing
a programmable interface for managing networking devices is a central function of network controllers. Due to the prevalence of Linux-based environments in these
data centers, network administrators' familiarity with the Linux networking stack provides is a possible interface for network management.

\par With this consideration, BISDN developed Basebox, an open source Software-Defined Network controller that interacts between the Linux kernel's networking
library and networking devices. This provides a familiar and stable API to configure ports, VLANs and routes, and the advantages of using tools like 
systemd-networkd and iptables.

\section {Aims and objectives}

% Importance of proposed research 
% Research aims

In a first stage, this thesis aims to plan and develop a system that exposes the physical topology of connections between switches and server, and monitoring
the port configuration and statistics. This management system aims to clearly display changes in configuration and behaviour of the networking infrastructure
connected to a SDN controller, which will allow system operators to maintain finer control over their resources. In this first part, we aim to define the 
components to implement this management system, from the necessary modifications to the existing controllers, to the Graphical User Interface that users will 
interact with. Typical management solutions like Icinga \footnote{\url{https://www.icinga.com/}}, Zabbix \footnote{\url{https://www.zabbix.com/}} or
Graphana \footnote{\url{https://grafana.com/}} display relevant metrics for monitoring of network devices, including the display of the
physical topology, monitor the port status, display devices temperature, among others. In this stage, we define the Minimum Viable Product (MVP) for this system as a
simple Graphical User Interface that can display the connections between switches and servers, and expose port statistics like the number of received packets
and bytes, the number of errors.

\par To support these features, we propose an Operations Support System (OSS) that interacts with the controllers and implements intelligence for monitoring 
the state of the network. Due to data center's traffic profile, one common challenge for optimizing the networks resource utilization is the asymmetry of traffic, 
where most of the networking flows are short-lived, latency-sensitive quick bursts of packets, but do not
amount for the total traffic in the network. The main contributor for the total traffic volume are the large and long-lived flows, usually called elephant flows.
Solution to this problem will increase the networks capability of splitting resources between the multiple competing virtualised applications, and, as a consequence,
their associated flows. Furthermore, by maintaining a system that monitors and alarms network operators of the occurrences of large data streams, this will provide
insight to the network operators to plan ahead their network resources. To implement this functionality, we researched the most common ways that detection of network
abnormalities are done, and propose an algorithm that enables quick detection of changes in the network traffic behaviour via simple statistical methods.

\section{Organization}

% Order of the document

This paper is divided in the following sections:

\begin{itemize}
    \item \textbf{Background} begins with an overview of Software-Defined Networking, providing an insight on the protocol behind these environments,
and presenting information on the existing applications. We also give an overview of network management, and approach what the roles of management systems have
in current data centers. We then provide a description on the architecture of the system this thesis will integrate.

    \item \textbf{Related Work} explores the deeper concepts of statistical detection, providing a formal framework for change detection mechanisms. We approach
the techniques traditionally used in change detection mechanisms, and evaluate the performance measures used to assess their efficiency. We also explore further
the concept of elephant flows, exploring a possible mathematical representation, and present the mechanisms typically used for detection and mitigation of this 
network phenomena.

    \item \textbf{Monitoring SDN switches} is a chapter dedicated to defining the proposed architecture for this thesis, by defining the components and the high
level approaches for implementing this solution.

    \item \textbf{Management API} approaches the first stage of the developed work. First, we design the networking entities with a set of standardized data models
for networks, so that the structure for creating the management interface is logically organized. Then we explore the researched alternatives for implementing
the transport protocol between the network controllers and the Graphical User Interface. Finally we demonstrate the final Graphical User Interface developed for 
this stage. 

    \item \textbf{Elephant Flow Monitoring} is the final chapter, where we explore the proposed solution for the OSS, more specifically the part related to the 
elephant flow monitoring. In this chapter we explore the design of the algorithm for monitoring the port changes, using the statistical methods presented in 
the Related Work chapter. Finally, we present the results of the detection algorithm, and propose a set of optimizations for use in this part.
\end{itemize}
