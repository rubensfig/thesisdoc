\chapter{Introduction} \label{chap:introduction} 
\hspace{0.5cm}

%%Era uma vez, ...

\section {Context} \label{context}
\hspace{0.5cm} 

%%  BISDN has implemented a managed WiFi network in a field trial. On the road to productisation, one of the major problems is the redundant and secured session information storage. The work proposed by the candidate will be evaluated within the existing trial system. 

Public and private access to the internet is a basic necessity in the modern society. In a modern setting, the rising requirements that exist in supporting users with mobile devices and their own computers must be met, in order to provide a stable and secure platform to ensure connectivity across different locations, devices, and more. In this context, managed Wi-Fi networks are defined as a way to deploy networks that are optimized in order to meet the specific requirements of users/ applications. 

%%QUOTE THE MARKET REACH
\par As we see the proliferation of these systems, due to increasingly demanding environments, the market for managed Wi-Fi systems is planned to reach \$1.7 billion in 2018. This raise in market is related to the distribution of the enterprises, where they are composed of several smaller locations, often lacking specialized IT staff that have strong networking capabilities, or even just the infrastructure is not suitable to installing these complex systems.
%%REDO

\par The creation of the As-A-Service model comes in consequence of these previous factors. In this specific case, the development of managed Wi-Fi as a system following the Infrastructure-As-A-Service (IAAS) mentality has allowed for the removal of investment in large WLAN infrastructure, related not only to the maintenance, but also the planning, operation, and installation of the service, allowing to transfer the cost to a trusted third-party. 

\par The previous paragraph shows a brief introduction to the motivation that lead to several companies to try to develop and support these sort of services. One of them, and the main proponent of the theme of the dissertation is Berlin Institute for Software Defined Networks - BISDN, and this research integrates a trial system that is being implemented, and we aim to solve one of the problems they faced during the road to productisation.
 
\par This document serves as an introduction to the future work to be developed in the dissertation. The organization is this report is as follows: in chapter 2, the state-of-the-art of similar mechanisms and a bibliographic revision is done; in chapter 3, the specific details of the problem are presented, and we start detailing some aspects that should be featured during the development of this dissertation; then we try to define a possible work plan, by defining a timeline, and approach to the development of the final product.

%% Continue 

\section {Motivation}
\hspace{0.5cm} 

By analyzing the current market offerings of large scale Wi-Fi solutions, we can see that the most offerings have a non ideal solution to cover the need of constant access to the internet, either by supplying a system that is not well optimized to cover the entire use cases, or by providing a system that isn't scalable, of fixed configuration, and that needs large, specialized IT teams that aren't always available due to several constraints. 
\par The issue of user information, in the scope of these networks, is a sensitive one. The storage and manipulation of this data should be used in the most secure way, to safeguard their privacy, and there should be safeguards in order not to lose the entire data. 
\par In this case, the theme of the future dissertation is proposed, to develop a tool that, in integration with larger systems, can provide a way to solve this problem.

\section {Goals}
\hspace{0.5cm} 

%% The main challenge explored is the storage of secure and redundant session information, in order to provide a way for users to interact with network in a seamless and reliable manner.
%%Seeing the evolution of WLAN infrastructure evolve into being a standard way of deploying large scale access networks in 

During the development of this dissertation, we aim in exploring the possibility of using cloud technologies to store, in a secure and redundant way, customer configuration and session data, to be used while deployed into large scale managed Wi-Fi systems already in testing. Our main goal, should feature an analysis of available cloud database solutions, and choose the better platform to suit our needs. It should also feature an in depth analysis of session information in a Wi-Fi managed network. By combining the two, we aim to propose a solution for a problem in deploying public Wireless LAN, the management of several users.
 