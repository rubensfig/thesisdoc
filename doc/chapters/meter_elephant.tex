\chapter{Elephant Flow Monitoring} \label{chap:me} 

\section {Testing Environment}

The design of a testing environment must allow for the accurate simulation of the traffic conditions on the real DCNs, and should provide the flexibility to 
understand and change the underlying topologies. There requirements clearly indicate a strong motivation for deploying a testing environment in a virtualised
environment, using tools like \textit{mininet}, which provides a miniature network that can be changed as needed. This testing suite provides a strong 
alternative to deploying these changes in hardware.

\par Despite the developments previously made to the SDN controllers, utilizing these in combination with the virtualised environment poses a challenge, related 
to the implementation of the OpenFlow protocol in the hardware and software switches. Hardware switches that were used for testing in the implementation of
the GUI have a modified version of the OpenFlow tables structure, OFDPA \footnote{OpenFlow Data-Path Abstraction}, and the libraries that make up the 
controller are designed around this. To utilize the controllers, changes to OpenvSwitch would be required, or an alternative would have to be 
discovered, which would limit functionality of the controllers.

\par To solve this issue, and to have a stable controller, that works as intended, for this step we adopt a different controller, and focus on the mechanisms that 
are also present in the hardware controllers. Furthermore, researching other approaches provides ideas that can later be adopted in these. The chosen controller
was Floodlight \footnote { XXX - insert link here}, for it is continuously updated, and provides a REST API for obtaining statistics, setting table rules.

\par These elements compose the testing environment that can be seen in \ref{fig:test_setup}. To ease the installation of the utilized applications, 
we based these applications on VMs and containers, and the installation files can be found on the following page \url {example.com}.

\pagebreak
% XXX - Make a testing setup picture
\begin{figure} 
    \centering
    \includegraphics[width=0.4\textwidth]{}
    \caption {PLACEHOLDER}
    \label{fig:test_setup}
\end{figure} 

\subsection {Performed tests}

\subsubsection {}


\section {Algorithm design}
\subsection {Formal definition}


\par $B_{XX}$ and $P_{XX}$ describe to the port statistics obtained from the controller, the byte and packet counters, respectively, and the indexes describe if the counters account for the transmitted or received data in that port.
\par A time series $y_t$ can be decomposed in the following parameters: $S_t$ which accounts for the seasonal component of the time series data; $T_t$ accounting for the constant trend in the data, and $R_t$, which accounts for the 
residuals. These can be combined to the following equation:

\section {Testing setup}
\section {Tests}
\section {Results}

