\chapter{Elephant Flow Monitoring} \label{chap:me} 

\section {Testing Environment}

The design of a testing environment must allow for the accurate simulation of the traffic conditions on the real DCNs, and should provide the flexibility to 
understand and change the underlying topologies. There requirements clearly indicate a strong motivation for deploying a testing environment in a virtualised
environment, using tools like \textit{mininet}, which provides a miniature network that can be changed as needed. This testing suite provides a strong 
alternative to deploying these changes in hardware.

\par Despite the developments previously made to the SDN controllers, utilizing these in combination with the virtualised environment poses a challenge, related 
to the implementation of the OpenFlow protocol in the hardware and software switches. Hardware switches that were used for testing in the implementation of
the GUI have a modified version of the OpenFlow tables structure, OFDPA \footnote{OpenFlow Data-Path Abstraction}, and the libraries that make up the 
controller are designed around this. To utilize the controllers, changes to OpenvSwitch would be required, or an alternative would have to be 
discovered, which would limit functionality of the controllers.

\par To solve this issue, and to have a stable controller, that works as intended, for this step we adopt a different controller, and focus on the mechanisms that 
are also present in the hardware controllers. Furthermore, researching other approaches provides ideas that can later be adopted in these. The chosen controller
was Floodlight \footnote { XXX - insert link here}, for it is continuously updated, and provides a REST API for obtaining statistics, setting table rules.

\par These elements compose the testing environment that seen in \ref{fig:test_setup}. To ease the installation of the utilized applications, 
we based these applications on VMs and containers, and the installation files can be found on the following page \url {example.com}.

\pagebreak
% XXX - Make a testing setup picture
\begin{figure} 
    \centering
    \includegraphics[width=0.5\textwidth]{meter_eleph/testing_setup}
    \caption {The high level overview of the testing setup}
    \label{fig:test_setup}
\end{figure} 

\par Figure \ref{fig:test_setup} describes the testing setup designed for testing during this dissertation. This setup provides a close approximation to setups used in the edge layers of data centers, and keeping the
resource consumption of the virtualised network and controller to a minimum. In the diagram, the hosts are shown using the H\textsubscript{X} notation, ranging from 1 to 4, and the switches use the S\textsubscript{X} 
notation. Information about the hosts, like the IP and MAC addresses can also be seen, and the port numbers used are also displayed.

\section {Algorithm design}

\begin {equation*}
\centering
x_i = 
\begin{bmatrix}
B_{RX}\\
P_{RX}\\
B_{TX}\\
P_{TX}\\
\end{bmatrix}
\end {equation*}

\par $B_{XX}$ and $P_{XX}$ describe to the port statistics obtained from the controller, the byte (B\textsubscript{XX})and packet (P\textsubscript{XX}) counters, respectively, and the indexes describe if the counters
account for the transmitted or received data in that port.

\par Using the techniques presented in section \ref{subsec:change_detection}, we build algorithm \ref{alg:high_level}. This is a high level view of the steps required to obtain the detection, and in the following sections,
we introduce each step, and provide some clarifications on the design decisions.

\pagebreak

\begin{algorithm}[H]
    \caption{Elephant Detection Algorithm - High Level} \label{alg:high_level}
    \begin{algorithmic}[1]
        \Procedure {Elephant Flow Detection}{}
            \State Initialization
            \State Query controller
            \Loop
                \State Error calculation
                \State Prediction
                \State Detection
                \If {Detection}
                    \State Raise Alarm
                \EndIf
            \EndLoop
        \EndProcedure 
        \State \Return Alarm Times
    \end{algorithmic}
\end{algorithm}

\subsection{Initialization}

The initialization step of the algorithm is a crucial step for obtaining correct results in the algorithm. This step allows for the correct initialization of the model parameters, including the trend component, and to provide a 
baseline for the expected traffic on the network. Due to these factors, it is assumed that no traffic abnormalities exist during this stage, but a longer period for initialization can account for short bursts of higher traffic.

\begin{algorithm}[H]
    \caption{Elephant Detection Algorithm - Initialization} \label{alg:high_level}
    \begin{algorithmic}[1]
        \Procedure {Initialization}{}
            \State initialization period = 30s
            \While {t <= initialization period}
                \State x = Query controller
                \State initialization measures += x
                \State Determine prediction
            \EndWhile
            \State Linear Regression (initialization measures)
        \State \Return Linear regression coefficient
    \end{algorithmic}
\end{algorithm}

\par Detrending the data using the linear regression function comes from the fact that on the end of the initialization period, plotting the received packets, bytes results in figure \ref{fig:init_plot}.

% Display initial graph of Received Bytes
\begin{figure} 
    \centering
    \includegraphics[width=0.5\textwidth]{}
    \caption {Plotting the initial measurements of B\textsubscript{RX}}
    \label{fig:test_setup}
\end{figure} 

\section {Testing setup}
\section {Tests}
\section {Results}

