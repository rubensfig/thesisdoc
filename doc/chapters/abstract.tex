\chapter*{Resumo}
%\addcontentsline{toc}{chapter}{Resumo}
Os requisitos crescentes dos serviços em nuvem de hoje requerem a evolução da infraestrutura de rede para suportar a quantidade crescente de dados que são
processados todos os dias. Isso significa que os operadores de centros de dados devem projectar ou adaptar os seus ambientes de rede em nuvem para fornecer
uma conexão estável e confiável.  Uma infraestrutura otimizada muitas vezes significa também a redução de custos na utilização dos sistemas e nos gastos em energia.

\par À medida que as redes crescem e são mais complexas, os sistemas devem ser implementados de forma a permitir não só acompanhar de perto os recursos que a compõem 
 mas também permitindo uma certa liberdade para a possível evolução dos requisitos. Como tal, as soluções típicas dos fornecedores não se encaixam
realmente nessa paisagem de constante mudança, uma vez que apresentam soluções muito sólidas e verticalmente integradas. O paradigma do Software Defined Networking, 
no entanto, é capaz de resolver esse problema, pois permite o controlo centralizado das redes subjacentes, proporcionando visibilidade e controlo sobre os 
dispositivos, simplificando o diagnóstico de erros e proporcionando uma maior capacidade de resposta.

\par Neste trabalho, propomos um sistema de gerenciamento modular para controladores de Software Defined Networks dos centros de dados da nuvem, fornecendo aos
administradores de sistemas uma plataforma simples, destacando-se e ressaltando a visualização da topologia de rede e monitorização das portas dos dispositivos. A
modularidade também fornece uma plataforma simples para estender a funcionalidade dos controladores de rede, que podem ser usados para implementar a detecção de 
anormalidades e optimizar os caminhos de encaminhamento de fluxo, entre outros.

\chapter*{Abstract}
%\addcontentsline{toc}{chapter}{Abstract}
%\https://www.th-wildau.de/fileadmin/dokumente/studiengaenge/europaeisches_management/dokumente/Dokumente_EM_Ba/Abstracts_in_English.pdf

The rising requirements of today's cloud services require the evolution of networking infrastructure to support the increasing amount of data that is processed
every day. This means that data center network operators must design or adapt their cloud networking environments to provide a stable and reliable connection.
Better optimized infrastructure often also means cost reductions in network utilization and energy savings.

\par As networks grow larger and more complex, systems must be put in place that allow for closely monitoring the resources that make up the network, while also 
allowing for a certain freedom for the possible constant change of the network. As such, typical vendor solutions don't really fit into this ever changing landscape,
since they present very solid and vertically integrated solutions. The Software Defined Networking paradigm, however, is able to solve this issue, since it enables
the centralized control of the underlying networks, providing visibility and control over the network's devices, simplifying error diagnosis and troubleshooting. 

\par In this work we propose a modular management system for cloud data center Software Defined Networking controllers, providing system administrators a simple
platform to view their network's topology, monitor networking devices ports, etc. The modularity also provides a simple platform to extend the functionality 
of the networking controllers, that can be used to implement detection of network abnormalities and optimize flow forwarding paths, among others.
