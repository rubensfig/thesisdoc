\chapter*{Resumo}
%\addcontentsline{toc}{chapter}{Resumo}

\chapter*{Abstract}
%\addcontentsline{toc}{chapter}{Abstract}
%\https://www.th-wildau.de/fileadmin/dokumente/studiengaenge/europaeisches_management/dokumente/Dokumente_EM_Ba/Abstracts_in_English.pdf

The rising requirements of today's cloud services require the evolution of networking infrastructure to support the increasing amount of data that is processed
every day. This means that data center network operators must design or adapt their cloud networking environments to provide a stable and reliable connection.
Better optimized infrastructure often also means cost reductions in network utilization and energy savings.
% \par As networks grow larger and more complex, systems must be put in place that allow for closely monitoring the resources that make up the network, while also 
% allowing for a certain freedom for the possible constant change of the network. As such, typical vendor solutions don't really fit into this ever changing landscape,
% since they present very solid and vertically integrated solutions. The SDN paradigm, however, is able to solve this issue, since it enables for the centralized
% control of the underlying networks, which provides visibility and even control over the network, simplifying network diagnosis or troubleshooting. 
