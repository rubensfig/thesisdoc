\chapter{Conclusion}

\section{Summary of Results}

The main objective for this thesis was building a management system that would integrate with a pre-existing Software-Defined Network controller, exposing information
for network operators to manage and configure their networking infrastructure. With these requirements, we have built a management environment that extends the
previously existing system, by adding a modular interface to baseboxd and CAWR, allowing further developments in the field of Traffic Engineering in the Basebox
environment.  Complementing this system, we have also designed a Graphical User Interface for interaction with the users, allowing for simple visualisation of the
network's physical topology, and the display of interfaces' statistics, like the packets and bytes received and sent, or the number of errors. 

\par We have also proposed an algorithm that allows for monitoring traffic changes in ports, in order to detect elephant flows in the network. Despite not having
used the Basebox system for testing this algorithm, due to differences in the testing environment, we believe that the same algorithm can be used for large flow 
detection in the Basebox stack. We have shown that a simple method can be employed by operators to monitor the state of their network, and rely on this algorithm
to provide them with alarms of port changes.

\section{Future Work}

Despite our conclusion that the main objectives of the thesis were achieved, the large scope of themes that this topic encompasses means that not everything could be 
successfully covered. As for more immediate concerns, the next steps in guaranteeing a stabler product would be to expand the GUI to report and configure VLANs in 
each port that is monitored; and support layer 3 functionality, such as visualising next hop neighbours, routing tables, etc. In regards to longer term goals, 
continuing the work on monitoring not only the port change, but the actual flow that contributes to the largest changes in ports. This could be expanded into a
system that analyses the services and applications that contribute the most to the traffic volumes in the network, which can then be further optimized by reporting
the periodicity of the largest traffic volumes. The aspect of providing a system that can discriminate the traffic by transport protocol is also an interesting
research topic for providing Quality of Service techniques to Software-Defined Networks.
