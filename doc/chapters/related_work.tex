\chapter{Statistical Detection} \label{chap:stat_det} %% chapter 3

\section {Introduction}

As a result of the large scale of current data centers, maintaining control over these networks proves a difficult task, for example compared to more telecom networks. Networks operators must then adapt to the current situation
by improving the monitoring infrastructures to allow faster response to problems and possible anomalies.  Building a feature complete management API for a SDN controller means that information obtained from the port and 
flow statistics previously implemented should return information about the global state of the network, so that root-cause analysis of the source of network issues can be done faster and easier,
 which reflect on better service and lower costs for network operators. 
\par Network behaviour analysis is defined by the constant monitoring of a network, so that events that compromise the "healthy" state of the network can be removed or mitigated. These include not only cases where the anomalies are
caused with malicious intent, like the case of DDoS attacks, but also failure of network devices or changes in user behaviour \cite {traffic_anomaly_control_charts}. These systems are equipped with alarm capabilities, so that 
system administrators can quickly respond to changes, possibly even giving some information about the source of the problem. However, the automation of these monitoring processes means that the possible existence
of false alarms reduces the operators capabilities to act on actual failures. 
\par This chapter focuses on the recent research done in order to implement systems that rely on statistical analysis for monitoring the state of the networks and detection of abnormalities.

\section {Traffic anomaly detection}

Detection of network anomalies is subject to intense research, and as such, several methods were developed, that assume different levels of control over the network and provide different results to different applications. Our goal
in this section is then to provide a description of the different types of network issues that can occur, and some proposed solutions for these issues.
\par In regards to the detection of elephant flows, the usual mechanisms that are employed are \cite {CITE - http://shiftleft.com/mirrors/www.hpl.hp.com/personal/Praveen_Yalagandula/papers/INFOCOM11.pdf}:

\begin {itemize}
    \item By having the services/ server signal the controller that their flows are elephants prior to the transmission. This is the most accurate way to "detect" and employ QoS techniques, but since applications and services
need to support this feature this is not commonly used.
    \item Via sampling the packets on the network, and employing packet header tests, or packet signature analysis for detecting deviations from the standard network behaviour. This assumes that the knowledge of the state of the 
network under normal conditions is known, so that techniques like entropy 
\end {itemize}

\section {Control charts}
