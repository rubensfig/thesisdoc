\section {Traffic engineering in data centers}

Understanding the impact that of elephant flows in the normal operation of a data center requires understanding the traffic characteristics of the typical cloud 
data center. The geographical proximity and localization of large data centers optimizes the interoperability that applications may require by minimizing 
propagation delay that could be present if the links between servers was larger, however cloud data centers used for costumer faced applications and those employed
in data intensive tasks may present different requirements, which poses a problem in optimizing the network. Furthermore, absence of publicly available data
contributes to the challenge of researching data centers \cite{CITE - network_traffic_characteristics_wild.pdf}.

\par Typical cloud data centers operate at a ratio of 1:1000 staff members to servers \cite{CITE - cloud_dc_research_problems}, which points to an essential need for 
extensible automation and failure recovery plans for optimal operation. Automation is central for cost reduction strategies in data centers, reducing the 
impact of failures caused by human errors.

\par Cost management is also achieved by improving power consumption, which correlates with improved methods for balancing load on the servers 
\cite{CITE - performance_optimization_virtual_machine_placement}. Load balancing is the concept of moving the load of an overloaded server to an underutilized one,
which reduces performance degradation, and increase recovery from failures \cite{CITE - server_storage_virtualization}. In a virtualised environment the possibility
of moving Virtual Machines (VMs) across servers and racks facilitates distributing the load without downtime, but the migration decision is not trivial due to the
large amount of variables involved, for example, the bandwidth, memory and CPU that is available on each server supporting the migrated VM. Furthermore, if an
applications workload changes over time, one decision may not apply for further migration \cite{CITE - multi_objective_vm_cloud_dc}. Netshare
\cite{CITE - netshare_predictable_bw_allocation.pdf} proposes a system that optimizes bandwidth allocation by imposing max-min sharing on services, using a 
centralized controller for orchestration. 

\par In regards to network traffic \cite{mori_identifying_2004, benson_network_2010},

\begin{itemize}
    \item for cloud data centers, the majority of traffic is internal to each rack,
    \item link utilizations are low in aggregation and edge layers,
    \item 90\% flows are small and last hundreds of milliseconds, but total traffic volume is largely dominated by the remainder, called \textbf{elephant flows}.
\end{itemize}

\par 
