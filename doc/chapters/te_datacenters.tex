\section {Traffic engineering in data centers}

Understanding the impact of elephant flows in the normal operation of a data center requires understanding the traffic characteristics of the typical cloud 
data center. The geographical proximity and localization of large data centers optimizes the interoperability that applications may require by minimizing 
propagation delay that could be present if the links between servers was larger, however cloud data centers used for costumer faced applications and those employed
in data intensive tasks may present different requirements, which poses a problem in optimizing the network. Furthermore, absence of publicly available data
contributes to the challenge of researching data centers \cite{benson_network_2010}.

\par Typical cloud data centers operate at a ratio of 1:1000 staff members to servers \cite{greenberg_cost_2008}, which points to an essential need for 
extensible automation and failure recovery plans for optimal operation. Automation is central for cost reduction strategies in data centers, reducing the 
impact of failures caused by human errors.

\par Cost management is also achieved by improving power consumption, which correlates with improved methods for balancing load on the servers 
\cite{meng_improving_2010}. \textbf{Load balancing} is the concept of moving the load of an overloaded server to an underutilized one,
which reduces performance degradation, and increase recovery from failures \cite{singh_server-storage_2008}. In a virtualised environment the possibility
of moving Virtual Machines (VMs) across servers and racks facilitates distributing the load without downtime, but the migration decision is not trivial due to the
large amount of variables involved, for example, the bandwidth, memory and CPU that is available on each server supporting the migrated VM. Furthermore, if an
application's workload change over time, one decision may not apply for further migrations \cite{xu_multi-objective_2010}. Netshare
\cite{radhakrishnan_netshare_2012} proposes a system that optimizes bandwidth allocation by imposing max-min fair sharing on services, using a 
centralized controller for orchestration. In \cite{shrivastava_application-aware_2011}, a minimization approach is applied to the migration problem, by generating
a cost function considering the variables associated to VM placement, computing the impact of moving a certain VM to a physical host, and the migration destination
is selected with the least amount of generated impact.

\par \textbf{Caching} is act of duplicating content across a network, in order to optimize access to frequently accessed content, minimizing network congestion
at peak access hours \cite{maddah-ali_fundamental_2014}. For Software Defined Networking, the centralized controller provides an optimal environment to implement caching 
in data centers due to high level knowledge of the tenants in the network. Moirai \cite{stefanovici_software-defined_2015} presents a programmable data-plane caching 
system, that allows to prioritize workloads, providing per-workload bandwidth guarantees. In \cite{georgopoulos_cache_2014}, the last mile delivery of Video-On-
Demand problem is solved with the inclusion of OpenFlow to create dynamic caches using hardware independent statistics, and provide support for additional policies
like load balancing. These systems show an evolution from historical caching mechanisms, that do not tend to support operations necessary in data center network 
operations, with higher bandwidth and storage requirements \cite{stefanovici_software-defined_2015}.

\subsection {DCN Traffic}

\par The several studies proposed in traffic engineering for traditional networks do need to be revised in DCN's, since metrics like propagation delay can be
negligible, due to the physical proximity of nodes in DC's \cite{bari_data_2013}. However, research in this topic can be a difficult task,
since many data center operators do not publish information about their applications and services.

\par By collecting data from different types of DC's, several studies have been made about the traffic characteristics 
\cite{benson_network_2010, benson_understanding_2009, roy_inside_2015}:

\begin {itemize}
    \item The placement of VMs and servers effects the bandwidth and link capacity, due to the variety of applications that can be running on the servers at any time,
        and this non-uniform placement of VMs contributes to higher amounts of traffic originating from the same rack
    \item The majority of flows \footnote {flows are sequences of packets sent from a source to a certain destination, either host, anycast or multicast domain} 
        are described as being small in size, and short in duration, which are usually described as \textit {mice} flows. 
        The counterpart to these are the \textit {elephant} flows, which occupy a very large share of the bandwidth, and degrade application performance, due to a
        choking effect to the latency-sensitive mice flows. 
        Applications are tied to the type of traffic they generate, where online gaming, VoIP and multimedia broadcasting usually originate mice flows, where the 
        large data transfers and file-sharing originate in elephant flows. Despite 90\% flows are small and last hundreds of milliseconds, total traffic volume is 
        largely dominated by the remainder, called \textbf{elephant flows} \cite{benson_network_2010}
    \item In a normal situation, link utilization is low in the layers apart from the core switches. In addition to this discovery, losses are associated with 
        spikes in traffic, instead being related to high utilization  of the link, which is one of the effects of the previously mentioned elephant flows.
\end {itemize}

\par Software Defined Networking based monitoring allows to increase the capability of conducting traffic monitoring and measurements. OpenTM 
\cite{krishnamurthy_passive_2010} utilizes the monitoring information to build the Traffic Matrix (TM) of an OpenFlow network by employing different methods of 
querying the switches, allowing to reduce the load associated with these queries and taking into account the different processing power of each device. Another 
method for estimating the link utilization is present in FlowSense \cite{yu_flowsense:_2013}, in which PACKET\_IN and FLOW\_REMOVED OpenFlow messages 
are monitored since these messages carry information of arrival of new flows and expiration of flow entries, providing a zero-overhead monitoring technique. This 
method has optimal performance when the flows in the network are short lived, posing worse performance as the flow time increases.

\subsection {Elephant flows} \label{sec:el_fl}

Detection of network anomalies is subject to intense research, and as such, several methods were developed, that assume different levels of control over the network 
and provide different results to different use cases. In this section we focus on the available techniques for detection and mitigation of large flows.

\subsubsection{General formulation of elephant flows}

Networks can be defined as an undirected graph $G(V, E)$, which contains a certain multicast group $M \subseteq V$, and a multicast tree $T \subseteq E$ 
\cite{lorenz_optimal_2003}. Considering the nodes $u, v \in M$, and the unique paths $p_{u \to v}$, an approach to define elephant flows
is to consider the optimal Quality-of-Service (QoS) partition of the unicast path $p$, or the multicast tree $T$ such as in 
\cite{lorenz_optimal_2003, lorenz_optimal_2002}. 

\par Considering the set of flows $F$ on the path $p$,

\begin{equation*}
    \centering
    F = <F_1, F_2, ..., F_n>,
\end{equation*}

\par and an arbitrary Quality of Service function $q()$, then $x_p = q(F)$ is the vector of QoS requirements 
for path $p$. In \cite{ros-giralt_mathematical_2017}, the optimal partition is the configuration that improves the QoS in $p$, or, in a formal
definition:

\begin{equation*}
    \centering
    q(F'_1, F'_2, ... F'_n) \geq q(F_1, F_2, ..., F_n),
\end{equation*}

\par where $F'_{X}$ is the partitioned set of flows. $q()$ is a generic function, since QoS can be classified with several parameters, like bandwidth, latency,
error rate, and so on, which must then be adapted from case to case. These parameters can be classified in two classes: \textit{bottleneck} and \textit{additive} 
\cite{lorenz_optimal_2003}. As previously mentioned, the main effect of elephant flows is the choking effect of the bandwidth
of the shorted lived flows, making the bandwidth a proper metric for usage in $q()$. Considering bandwidth as the global QoS requirement $Q$, the partition $x'_p$ 
must satisfy the condition

\begin{equation*}
    \centering
    x'_p = Q_{F \in p},
\end{equation*}

since bandwidth can be classified as a bottleneck parameter, where the link with the lowest bandwidth sets the total for the link. Considering the additive case, 
where the delay of the links can be used as an example, the total delay of the links will be the sum of the delays of each link.

\par Using this notation, $<F_e, F_m>$ identifies the optimal QoS partition where $F_e$ is the set of elephant flows, and $F_m$ is the set containing 
mouse flows. 

\subsubsection{Detection}

The problem of detection of traffic anomalies has been subject to extensive research, and several different approaches have been proposed. These methods are
based on different techniques \cite{curtis_mahout:_2011}:

\begin {itemize}
    \item Modifications to the applications and services to notify the controller about the state of the traffic on each service. Despite this approach resulting
         in the most accurate "detection" of network anomalies, support for this technique is not extensive, due to the required changes to each service, 
         and it does not account for abrupt changes on the traffic
    \item By setting hard limits on the transmission capabilities of each port and switch, the controller is ensured of the non existence of flows that could 
        impact network performance. This is a mitigation strategy that does not scale well to very large networks, since it requires the storage of the rules
        imposed to every port, and can potentially lead to the inefficient use of network resources, reducing flexibly on the DCN's.
    \item By employing sampling and collection techniques, using mechanisms like sFlow (see \ref{sec:sflow}), and building the profile of the normal state of 
        the network,
        this method can detect outliers that deviate from the normal state of the network. Utilizing this method reduces impact that continuously polling the network
        might have, while reducing impact on the packet and byte counts \cite{brauckhoff_impact_2006}, but 
        the loss of information inherent to sampling may be a challenge on successful deployment, which should account for optimal sampling 
        strategies and inference from the obtained statistics.
    \item Periodic polling of the statistics from the switch, and employing statistical analysis methods to determine change points in the state of the network.
\end {itemize}

\par Mahout \cite{curtis_mahout:_2011} presents a system that allows for elephant flow detection by monitoring end hosts. Via implementation of a shim layer on top
of every host present in the network, the proposed system allows to tag the traffic that belongs to larger flows, reducing the complexity that is generated 
if the monitoring was done at the aggregation or core switches. Detection itself is done by comparing the number of bytes in a buffer to a certain threshold.
In Hedera \cite{al-fares_hedera:_2010} the problem of maximizing network bandwidth is approached, via detection of large flows, and optimizing the placement on the switches
according to the demands of these flows. A sampling approach for detection of network anomalies is explored in \cite{jun_ddos_2014}, where the 
proposed method relies on analysing the sampled packets and building a tuple of the traffic characteristics, like the source and destination IP, source and 
destination ports, and the protocol. Using machine learning methods, this system then calculates the entropy of each field to a certain threshold, which
allows for detection and classification of different types of DDoS attacks. In TE, the use of entropy has been vastly discussed \cite{brauckhoff_impact_2006,
lall_data_2006}, since it can be used in fine grained traffic engineering systems, as well as anomaly detection systems, since entropy based methods
detect the changes in traffic distribution, and changes that would be small enough for volume change detectors are visible in these systems. Furthermore, traffic
classification is now possible, since similar changes in traffic distribution cause the same changes in entropy, and this factor can be used in classifying the 
different types of DDoS attacks, for example.

\par In the case of systems that explore the changes in packet and bytes, \cite{munz_traffic_2010} proposes a simple approach
for traffic change detection, based in time series and Principal component analysis (PCA). By measuring the state of the network, using the links, flows and packets
statistics, a statistical approach to finding the time locations of change points in the network traffic is designed, by combining different types of statistical
process control techniques.
PCA removes the correlation of a set of observations, separating the seasonal variation from the residual variability. Another system exploring PCA for anomaly
detection is studied in \cite{lakhina_characterization_2004}, where traffic flow measurements from Origin-Destination (OD) links are used to determine the anomalies
present in the link. 

\subsubsection{Mitigation}

Now that the methods for identifying elephant flows were presented, discussion of mitigation techniques to decrease the impact on the smaller, latency sensitive
flows must be addressed. 

\par In general, post-detection actions for solving elephant flows can be based in \cite{pettit_open_2014, noauthor_mice_2013}

\begin{itemize}
    \item separation the queues from elephants and mice flows,
    \item routing the elephants on a separate path or forwarding them in separate networks,
    \item separation the elephants into smaller flows, using different ports, and rely on reliable transmission mechanisms, such as those present on TCP
        to organize and re-order the packets.
\end{itemize}

\par A novel approach for mitigation of elephant flows is present in \cite{zhu_intelligent_2015}, where the timeout present in OpenFlow rules in presented as 
a dynamic way to manage the control plane load and improve the usage of flow table resources, managing the flows based in their inter packet arrival interval and
periodicity. Basic functionality of this system is the management of the timeout setting of each flow, since, if this is proven to be too long for flows with 
long inter packet arrival periods, the flow rules on the switch could become out dated, leading to a possible switch table overflow; and if the timeouts are 
too short, the control plane can be overloaded, causing possible controller failures.

\par Another solution proposed for managing elephants are present in congestion-aware systems, where the routing or forwarding decisions are done with a dynamic
overview of each links properties. The proposals in congestion-aware systems are relevant, due to the inefficiency of Equal-cost multi-path (ECMP) routing in data
centers, which causes low throughput and bandwidth utilization for elephant flows, and increases latency for mice flows \cite{wang_expeditus:_2017}. In 
\cite{wilson_better_2011}, link capacity is managed per flow by analysing the \textit{flow deadline} \footnote{Flow deadline is defined as the time that the flow has
to finish transmission, usually due to Service Level Agreements (SLAs)}. The reasoning behind analysis of the flow deadline is due to splitting resources on the link.
If the link's capacity is shared equally between every flow, then every flow will arrive at the same time, but likely the deadline will have been missed by some. 
Furthermore, consideration of the flow implies that every packet must arrive before the deadline. In the case of elephant flow mitigation, assigning larger portions
of the link's capacity to the shorter flows can be a possible strategy. In \cite{mittal_timely:_2015} presents a similar approach, instead analysing Round-Trip Time
(RTT), since this metric directly reflects the end-to-end latency of a link.
