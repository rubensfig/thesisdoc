%% artigo-exemplo.tex

\documentclass[a4paper]{IEEEtran}
\usepackage[utf8]{inputenc}
%\usepackage{latin}
\markboth{}{}
\usepackage[portuges]{babel}

\ifCLASSINFOpdf
  \usepackage[pdftex]{graphicx}  
\else
  \usepackage[dvips]{graphicx}
\fi

\renewcommand{\footnoterule}{\noindent\rule{0.5\columnwidth}{0.5pt}\vspace*{3pt}}

\begin{document}

% Título (usar \\ para quebra de linha)
\title{API design and implementation of a management interface for SDN whitebox switches}

% author names and affiliations
% use a multiple column layout for up to three different
% affiliations
\author{\IEEEauthorblockN{Rubens Jesus Alves Figueiredo $^$}%
%\thanks{r.figueiredo.52@gmail.com}
\IEEEauthorblockN{Dr. Ana Cristina Costa Aguiar $^$}%
%\thanks{$^\dag$Contacto orientador}
\IEEEauthorblockN{Dr. Hagen Woesner$^$}%
%\thanks{$^\ddag$Contacto co-orientador}
}
% make the title area
\maketitle

% \markboth{Uma parte}{Outra parte}

\begin{abstract}
The rising requirements of today's cloud services require the evolution of networking infrastructure to support the increasing amount of data that is processed
every day. This means that data center network operators must design or adapt their cloud networking environments to provide a stable and reliable connection.
Better optimized infrastructure often also means cost reductions in network utilization and energy savings.

\par As networks grow larger and more complex, systems must be put in place that allow for closely monitoring the resources that make up the network, while also 
allowing for a certain freedom for the possible constant change of the network. As such, typical vendor solutions don't really fit into this ever changing landscape,
since they present very solid and vertically integrated solutions. The Software Defined Networking paradigm, however, is able to solve this issue, since it enables
the centralized control of the underlying networks, providing visibility and control over the network's devices, simplifying error diagnosis and troubleshooting. 

\par In this work we propose a modular management system for cloud data center Software Defined Networking controllers, providing system administrators a simple
platform to view their network's topology, monitor networking devices ports, etc. The modularity also provides a simple platform to extend the functionality 
of the networking controllers, that can be used to implement detection of network abnormalities and optimize flow forwarding paths, among others.
\end{abstract}

\begin{IEEEkeywords}
Software-Defined Networking, Cloud Data centers, OpenFlow, Networks, SDN
\end{IEEEkeywords}

\section{Introduction}

\IEEEPARstart{T}{he} rapid expansion of the cloud computing environment in the previous decade is related to the increasing demand in computational power that 
applications like distributed databases or data analysis have. Public cloud solutions like Amazon's Web Services, Microsoft's Azure, or private solutions offered
through OpenStack provide a very large pool of resources for developers to deploy applications with ease. Through economies of scale, these vendors have
centralized their solutions in very large scale data centers, consolidating their operations in a single location, allowing increased performance of applications, and
easier maintenance. The offer of virtualisation solutions also contributes to the recent surge in popularity of these systems, due to the less spending in
operational costs and improved utilization of hardware \cite{sims_david_carousels_2011}. Subscription based systems are typically available for renting, allowing 
users to use services on a Virtual Machine.

\par Using open source applications and whitebox hardware has also contributed to the success of these environments, due to the possible cost reductions. Software
Defined Networking (SDN), has proven to be a reliable environment to manage data center environments, due to the centralization of the network
controllers, improved programmability of the network's data plane, and improved management systems. Network programmability, despite not being exclusive to the 
SDN framework, eliminates the effort in individually configuring every network device, which in large scale environments becomes an impossible task. This model
also provides the network engineers and software developers a high level network abstraction that is used to monitor network utilization and optimize resource
utilization.

\par To support monitoring in the SDN environment, we propose an Operations Support System (OSS) that interacts with the network controllers and implements 
intelligence for monitoring the state of the network. Due to data center's traffic profile, one common challenge for optimizing the networks resource utilization
is the asymmetry of traffic, where most of the networking flows are short-lived, latency-sensitive quick bursts of packets, but do not amount for the total traffic 
in the network. The main contributor for the total traffic volume are the large and long-lived flows, usually called elephant flows. By maintaining a system that
monitors and alarms network operators of the occurrences of large data streams, this will provide insight to the network operators to plan ahead their network 
resources.

\par This document is organized in:

\begin{itemize}
    \item \textbf{Related Work} is an overview of Software-Defined Networking, providing an insight on the protocol behind these environments,
and presenting information on the existing applications. This section also sums up a formal framework for change detection mechanisms.

    \item \textbf{} is an overview of Software-Defined Networking, providing an insight on the protocol behind these environments,
and presenting information on the existing applications. This section also sums up a formal framework for change detection mechanisms.
\end{itemize}

\section{Related Work}
\section{Proposed Architecture}
\section{Results}
\section{Conclusion}
\section*{Agradecimentos}

Sem exagerar.

% referências

\bibliographystyle{IEEEtran}
\bibliography{../doc/thesis_reference}

\end{document}


