%% artigo-exemplo.tex

\documentclass[a4paper]{IEEEtran}
\usepackage[utf8]{inputenc}
%\usepackage{latin}
\markboth{}{}
\usepackage[portuges]{babel}

\usepackage{algorithm}
\usepackage{algpseudocode}
\ifCLASSINFOpdf
  \usepackage[pdftex]{graphicx}  
\else
  \usepackage[dvips]{graphicx}
\fi

\graphicspath{{}} 

\renewcommand{\footnoterule}{\noindent\rule{0.5\columnwidth}{0.5pt}\vspace*{3pt}}

\begin{document}

% Título (usar \\ para quebra de linha)
\title{API design and implementation of a management interface for SDN whitebox switches}

% author names and affiliations
% use a multiple column layout for up to three different
% affiliations
\author{\IEEEauthorblockN{Rubens Jesus Alves Figueiredo $^$}%
%\thanks{r.figueiredo.52@gmail.com}
\IEEEauthorblockN{Dr. Ana Cristina Costa Aguiar $^$}%
%\thanks{$^\dag$Contacto orientador}
\IEEEauthorblockN{Dr. Hagen Woesner$^$}%
%\thanks{$^\ddag$Contacto co-orientador}
}
% make the title area
\maketitle

% \markboth{Uma parte}{Outra parte}

\begin{abstract}
The rising requirements of today's cloud services require the evolution of networking infrastructure to support the increasing amount of data that is processed
every day. This means that data center network operators must design or adapt their cloud networking environments to provide a stable and reliable connection.
Better optimized infrastructure often also means cost reductions in network utilization and energy savings.

\par As networks grow larger and more complex, systems must be put in place that allow for closely monitoring the resources that make up the network, while also 
allowing for a certain freedom for the possible constant change of the network. As such, typical vendor solutions don't really fit into this ever changing landscape,
since they present very solid and vertically integrated solutions. The Software Defined Networking paradigm, however, is able to solve this issue, since it enables
the centralized control of the underlying networks, providing visibility and control over the network's devices, simplifying error diagnosis and troubleshooting. 

\par In this work we propose a modular management system for cloud data center Software Defined Networking controllers, providing system administrators a simple
platform to view their network's topology, monitor networking devices ports, etc. The modularity also provides a simple platform to extend the functionality 
of the networking controllers, that can be used to implement detection of network abnormalities and optimize flow forwarding paths, among others.
\end{abstract}

\begin{IEEEkeywords}
Software-Defined Networking, Cloud Data centers, OpenFlow, Networks, SDN
\end{IEEEkeywords}

\section{Introduction}

\IEEEPARstart{T}{he} rapid expansion of the cloud computing environment in the previous decade is related to the increasing demand in computational power that 
applications like distributed databases or data analysis have. Public cloud solutions like Amazon's Web Services, Microsoft's Azure, or private solutions offered
through OpenStack provide a very large pool of resources for developers to deploy applications with ease. 

\par Using open source applications and whitebox hardware has also contributed to the success of these environments, due to the possible cost reductions. Software
Defined Networking (SDN), has proven to be a reliable environment to manage data center environments, due to the centralization of the network
controllers, improved programmability of the network's data plane, and improved management systems. Network programmability, despite not being exclusive to the 
SDN framework, eliminates the effort in individually configuring every network device, which in large scale environments becomes an impossible task. 

\par Due to data center's traffic profile, one common challenge for optimizing the networks resource utilization is the asymmetry of traffic, where most of the 
networking flows are short-lived, latency-sensitive quick bursts of packets, but do not amount for the total traffic in the network. The main contributor for the 
total traffic volume are the large and long-lived flows, usually called elephant flows. By maintaining a system that monitors and alarms network operators of the
occurrences of large data streams, this will provide insight to the network operators to plan ahead their network resources.

\par This document is organized in:

\begin{itemize}
    \item \textbf{Related Work} is an overview of Software-Defined Networking environments and presenting information on the existing applications. 
    \item \textbf{Proposed Architecture} details the proposed architecture for this work, and the technologies used for implementation;
    \item \textbf{Results} displays the developed Graphical User Interface, and the results of the change detection method implemented for monitoring elephant flows;
    \item \textbf{Conclusion} sums up the main contributions of this work.
\end{itemize}

\section{Related Work}

\par Mahout \cite{curtis_mahout:_2011} presents a system that allows for elephant flow detection by monitoring end hosts. Via implementation of a shim layer on top 
of every host present in the network, the proposed system allows to tag the traffic that belongs to larger flows, reducing the complexity that is generated if the 
monitoring was done at the aggregation or core switches. Detection itself is done by comparing the number of bytes in a buffer to a pre-defined threshold. In
Hedera \cite{al-fares_hedera:_2010} the problem of maximizing network bandwidth is via detection of large flows and optimization of the placement on the switches
according to the demands of these flows. A sampling approach for detection of network anomalies is explored in \cite{jun_ddos_2014}, where the proposed method 
relies on analysing the sampled packets and building a tuple of the traffic characteristics (source and destination IP, source and destination ports, and the 
protocol).

\section{Proposed Architecture}

\begin{figure}
    \includegraphics[scale=0.4]{../doc/figures/bisdn/prp_system_low_level}
    \caption{Proposed system}
    \label{fig:fig}
\end{figure}

\par We build an Operations Support System (OSS) that provides the basis for the development of applications that monitor the state of the network, using an
implemented API to obtain the relevant information. This system allows further separation of roles in the network, in contrast to a system where the controller would
gather the roles of managing and monitor the network’s status, increasing the load on the controllers. Furthermore, this architecture increases the modularity of 
the components, enabling hot-swapping different modules, and allows parallel development of different features in the monitoring and management stack.

\par We provide a proof-of-concept composed of two components: the first is a Graphical User Interface (GUI) providing an user friendly interface to display 
topology and statistics, and the second is an intelligent system enabling the detection of elephant flows. In this section we describe in a high level way the 
approaches for the development of this system.

\section{Results}

\subsection{Graphical User Interface}

The connection between both controllers to the GUI provides the view for both controllers, which means that CAWR will present the view for the physical switches, 
bonded ports and hosts, while the baseboxd only shows the giant switch created by CAWR. Analysis of the global view of the state of the
network, include the addition of displaying configured VLANs in each port, and even provide a way to configure these VLANS via a GUI. Interaction with the nodes is 
possible, and clicking on each provides an insight to the statistics related to the ports in that node.

\subsection{Elephant Flow Monitoring}

\begin{algorithm}[H]
    \caption{Elephant Detection Algorithm - High Level} \label{alg:high_level}
    \begin{algorithmic}[1]
        \Procedure {Elephant Flow Detection}{}
            \State Initialization
            \State Query controller
            \Loop
                \State Calculate prediction error
                \State Predict next values
                \State Detection
                \If {Detection}
                    \State Raise Alarm
                \EndIf
                \State wait 2 seconds
            \EndLoop
        \EndProcedure
       \end{algorithmic}
\end{algorithm}

\par The initialization step of the algorithm is a crucial step for obtaining correct results in the algorithm, since they allow the correct initialization of
the model parameters to provide a baseline for the expected traffic on the network. It is assumed that no traffic abnormalities exist during this stage.

\par Time series analysis can generate forecasts for future values, assuming the temporal behaviour is maintained for future observations. During the design phase 
of the algorithm, we selected the exponential smoothing technique, since this is a commonly used technique in the reviewed literature \cite{jasek_usage_2013, 
munz_traffic_2010}, and provides a generally simple way to generate forecasts based in historical data.

\par One of the analysed change detection methods is the CUSUM algorithm. This method is used for monitoring parameters of a sample, by monitoring deviations of the
observations according to a certain target value. Typical implementations of this algorithm are based in an offline approach, calculating the alarm times
with knowledge of the entire data set. The adaptation of the CUSUM algorithm for utilization as an online technique is based on a sliding window that
is updated with every new sample. Applying this method has the advantages of using the CUSUM algorithm without needing extensive changes, while also reducing the 
amount of memory needed to apply this method.

\section{Conclusion}
% referências
\par We have built a management environment that extends the previously existing SDN controllers, allowing developments in the field of Traffic Engineering in the 
Basebox environment. We have also designed a Graphical User Interface for interaction with the users, allowing for simple visualisation of the network’s physical 
topology, and the display of interfaces’ statistics, like the packets and bytes received and sent, or the number of errors.

\par We have also proposed an algorithm that allows for monitoring traffic changes in ports, in order to detect elephant flows in the network. We have shown that a 
simple method can be employed by operators to monitor the state of their network, and rely on this algorithm to provide them with alarms of port changes.

\bibliographystyle{plain}
\bibliography{thesis_reference}
%\PrintBib{../doc/thesis_reference}
%\begin{thebibliography}{100}

\bibitem{sims_david_carousels_2011} David Sims.
\newblock Carousel's {Expert} {Walks} {Through} {Major} {Benefits} of
  {Virtualization}, July 2011.

\bibitem{sampaio_energy-efficient_2015} Altino Manuel~Silva Sampaio.
\newblock {\em Energy-efficient and {SLA}-based {Management} of {IaaS} {Cloud}
  {Data} {Centers}}.
\newblock {PhD} {Thesis}, Universidade do Porto (Portugal), 2015.

\bibitem{bakhshi_user-centric_2017} Taimur Bakhshi.
\newblock {\em User-{Centric} {Traffic} {Engineering} in {Software} {Defined}
  {Networks}}.
\newblock {PhD} {Thesis}, University of Plymouth, 2017.

\bibitem{open_networking_foundation_sdn_2014} {Open Networking Foundation}.
\newblock {SDN} {Architecture} {Overview}.
\newblock Technical TR\_SDN ARCH Overview 1.1 1111 2014, Open Networking
  Foundation, November 2014.

\bibitem{shin_software-defined_2012} Myung-Ki Shin, Ki-Hyuk Nam, and Hyoung-Jun Kim.
\newblock Software-defined networking ({SDN}): {A} reference architecture and
  open {APIs}.
\newblock pages 360--361. IEEE, October 2012.

\bibitem{open_networking_foundation_openflow_2015} {Open Networking Foundation}.
\newblock {OpenFlow} {Switch} {Specification} {Version} 1.3.5 ( {Protocol}
  version 0x04 ), March 2015.

\bibitem{broadcom_corporation_openflow_2017} {Broadcom Corporation}.
\newblock {OpenFlow}{\texttrademark} {Data} {Plane} {Abstraction} ({OF}-{DPA}):
  {Abstract} {Switch} {Specification}, January 2017.

\bibitem{sezer_are_2013} Sakir Sezer, Sandra Scott-Hayward, Pushpinder~Kaur Chouhan, Barbara Fraser,
  David Lake, Jim Finnegan, Niel Viljoen, Marc Miller, and Navneet Rao.
\newblock Are we ready for {SDN}? {Implementation} challenges for
  software-defined networks.
\newblock {\em IEEE Communications Magazine}, 51(7):36--43, 2013.

\bibitem{raz_2014_2014} Danny Raz, International~Federation for Information~Processing, Institute
  of~Electrical {and} Electronics~Engineers, Communications Society, and
  Computer Society, editors.
\newblock {\em 2014 10th {International} {Conference} on {Network} and
  {Service} {Management} ({CNSM} 2014): {Rio} de {Janeiro}, {Brazil}, 17 - 21
  {November} 2014 ; [including workshop papers]}.
\newblock IEEE, Piscataway, NJ, 2014.
\newblock OCLC: 931888885.

\bibitem{nunes_survey_2014} Bruno Astuto~A. Nunes, Marc Mendonca, Xuan-Nam Nguyen, Katia Obraczka, and
  Thierry Turletti.
\newblock A {Survey} of {Software}-{Defined} {Networking}: {Past}, {Present},
  and {Future} of {Programmable} {Networks}.
\newblock {\em IEEE Communications Surveys \& Tutorials}, 16(3):1617--1634,
  2014.

\bibitem{vishnoi_effective_2014} Anilkumar Vishnoi, Rishabh Poddar, Vijay Mann, and Suparna Bhattacharya.
\newblock Effective switch memory management in {OpenFlow} networks.
\newblock pages 177--188. ACM Press, 2014.

\bibitem{phemius_disco:_2014} Kevin Phemius, Mathieu Bouet, and Jeremie Leguay.
\newblock {DISCO}: {Distributed} multi-domain {SDN} controllers.
\newblock pages 1--4. IEEE, May 2014.

\bibitem{khondoker_feature-based_2014} Rahamatullah Khondoker, Adel Zaalouk, Ronald Marx, and Kpatcha Bayarou.
\newblock Feature-based comparison and selection of {Software} {Defined}
  {Networking} ({SDN}) controllers.
\newblock pages 1--7. IEEE, January 2014.

\bibitem{berde_onos:_2014} Pankaj Berde, William Snow, Guru Parulkar, Matteo Gerola, Jonathan Hart, Yuta
  Higuchi, Masayoshi Kobayashi, Toshio Koide, Bob Lantz, Brian O'Connor, and
  Pavlin Radoslavov.
\newblock {ONOS}: towards an open, distributed {SDN} {OS}.
\newblock pages 1--6. ACM Press, 2014.

\bibitem{project_floodlight_floodlight_2017} {Project Floodlight}.
\newblock Floodlight {Is} an {Open} {SDN} {Controller}, 2017.

\bibitem{medved_opendaylight:_2014} Jan Medved, Robert Varga, Anton Tkacik, and Ken Gray.
\newblock {OpenDaylight}: {Towards} a {Model}-{Driven} {SDN} {Controller}
  architecture.
\newblock pages 1--6. IEEE, June 2014.

\bibitem{bierman_restconf_2017} Andy Bierman, Martin Bjorklund, and Kent Watsen.
\newblock {\em {RESTCONF} {Protocol}}.
\newblock Number 8040 in Request for {Comments}. RFC Editor, January 2017.
\newblock Published: RFC 8040.

\bibitem{akyildiz_research_2016} Ian~F. Akyildiz, Ahyoung Lee, Pu~Wang, Min Luo, and Wu~Chou.
\newblock Research challenges for traffic engineering in software defined
  networks.
\newblock {\em IEEE Network}, 30(3):52--58, May 2016.

\bibitem{mousavi_early_2015} Seyed~Mohammad Mousavi and Marc St-Hilaire.
\newblock Early detection of {DDoS} attacks against {SDN} controllers.
\newblock pages 77--81. IEEE, February 2015.

\bibitem{curtis_mahout:_2011} Andrew~R. Curtis, Wonho Kim, and Praveen Yalagandula.
\newblock Mahout: {Low}-overhead datacenter traffic management using
  end-host-based elephant detection.
\newblock In {\em {INFOCOM}, 2011 {Proceedings} {IEEE}}, pages 1629--1637.
  IEEE, 2011.

\bibitem{brauckhoff_impact_2006} Daniela Brauckhoff, Bernhard Tellenbach, Arno Wagner, Martin May, and Anukool
  Lakhina.
\newblock Impact of packet sampling on anomaly detection metrics.
\newblock In {\em Proceedings of the 6th {ACM} {SIGCOMM} conference on
  {Internet} measurement}, pages 159--164. ACM, 2006.

\bibitem{noauthor_recommendation_1992} Recommendation {X}.700 {MANAGEMENT} {FRAMEWORK} {FOR} {OPEN} {SYSTEMS}
  {INTERCONNECTION} ({OSI}) {FOR} {CCITT} {APPLICATIONS}, September 1992.

\bibitem{fedor_simple_1990} Mark Fedor, James~R. Davin, Martin~Lee Schoffstall, and Dr~Jeff~D. Case.
\newblock {\em Simple {Network} {Management} {Protocol} ({SNMP})}.
\newblock Number 1157 in Request for {Comments}. RFC Editor, May 1990.
\newblock Published: RFC 1157.

\bibitem{rose_structure_1990} Dr~Marshall~T. Rose and Keith McCloghrie.
\newblock {\em Structure and identification of management information for
  {TCP}/{IP}-based internets}.
\newblock Number 1155 in Request for {Comments}. RFC Editor, May 1990.
\newblock Published: RFC 1155.

\bibitem{rose_management_1990} Dr~Marshall~T. Rose and Keith McCloghrie.
\newblock {\em Management {Information} {Base} for network management of
  {TCP}/{IP}-based internets}.
\newblock Number 1156 in Request for {Comments}. RFC Editor, May 1990.
\newblock Published: RFC 1156.

\bibitem{schonwalder_overview_2003} J{\"u}rgen Sch{\"o}nw{\"a}lder.
\newblock {\em Overview of the 2002 {IAB} {Network} {Management} {Workshop}}.
\newblock Number 3535 in Request for {Comments}. RFC Editor, May 2003.
\newblock Published: RFC 3535.

\bibitem{popa_cost_2010} Lucian Popa, Sylvia Ratnasamy, Gianluca Iannaccone, Arvind Krishnamurthy, and
  Ion Stoica.
\newblock A {Cost} {Comparison} of {Datacenter} {Network} {Architectures}.
\newblock In {\em Proceedings of the 6th {International} {COnference}},
  Co-{NEXT} '10, pages 16:1--16:12, New York, NY, USA, 2010. ACM.

\bibitem{bisdn_gmbh_software_2017} {BISDN GmbH}.
\newblock Software - {Defined} {Cloud} {Networking}, 2017.

\bibitem{jasek_usage_2013} Roman Ja{\v s}ek, Anna Szmit, and Maciej Szmit.
\newblock Usage of {Modern} {Exponential}-{Smoothing} {Models} in {Network}
  {Traffic} {Modelling}.
\newblock In Ivan Zelinka, Guanrong Chen, Otto~E. R{\"o}ssler, Vaclav Snasel,
  and Ajith Abraham, editors, {\em Nostradamus 2013: {Prediction}, {Modeling}
  and {Analysis} of {Complex} {Systems}}, pages 435--444, Heidelberg, 2013.
  Springer International Publishing.

\bibitem{munz_traffic_2010} Gerhard M{\"u}nz.
\newblock {\em Traffic anomaly detection and cause identification using
  flow-level measurements}.
\newblock Number 2010,06 in Network architectures and services. Network
  Architectures and Services, Techn. Univ. M{\"u}nchen, M{\"u}nchen, 2010.
\newblock OCLC: 845870015.

\bibitem{shrivastava_application-aware_2011} Vivek Shrivastava, Petros Zerfos, Kang-Won Lee, Hani Jamjoom, Yew-Huey Liu, and
  Suman Banerjee.
\newblock Application-aware virtual machine migration in data centers.
\newblock In {\em {INFOCOM}, 2011 {Proceedings} {IEEE}}, pages 66--70. IEEE,
  2011.

\bibitem{georgopoulos_cache_2014} Panagiotis Georgopoulos, Matthew Broadbent, Bernhard Plattner, and Nicholas
  Race.
\newblock Cache as a service: {Leveraging} sdn to efficiently and transparently
  support video-on-demand on the last mile.
\newblock In {\em Computer {Communication} and {Networks} ({ICCCN}), 2014 23rd
  {International} {Conference} on}, pages 1--9. IEEE, 2014.

\bibitem{maddah-ali_fundamental_2014} Mohammad~Ali Maddah-Ali and Urs Niesen.
\newblock Fundamental limits of caching.
\newblock {\em IEEE Transactions on Information Theory}, 60(5):2856--2867,
  2014.

\bibitem{greenberg_cost_2008} Albert Greenberg, James Hamilton, David~A. Maltz, and Parveen Patel.
\newblock The cost of a cloud: research problems in data center networks.
\newblock {\em ACM SIGCOMM computer communication review}, 39(1):68--73, 2008.

\bibitem{xu_multi-objective_2010} Jing Xu and Jose A.~B. Fortes.
\newblock Multi-{Objective} {Virtual} {Machine} {Placement} in {Virtualized}
  {Data} {Center} {Environments}.
\newblock pages 179--188. IEEE, December 2010.

\bibitem{radhakrishnan_netshare_2012} Sivasankar Radhakrishnan, Rong Pan, Amin Vahdat, and George Varghese.
\newblock Netshare and stochastic netshare: predictable bandwidth allocation
  for data centers.
\newblock {\em ACM SIGCOMM Computer Communication Review}, 42(3):5--11, 2012.

\bibitem{benson_network_2010} Theophilus Benson, Aditya Akella, and David~A. Maltz.
\newblock Network traffic characteristics of data centers in the wild.
\newblock In {\em Proceedings of the 10th {ACM} {SIGCOMM} conference on
  {Internet} measurement}, pages 267--280. ACM, 2010.

\bibitem{krishnamurthy_passive_2010} Arvind Krishnamurthy and B.~Plattner, editors.
\newblock {\em Passive and active measurement: 11th international conference,
  {PAM} 2010, {Zurich}, {Switzerland}, {April} 7-9, 2010: proceedings}.
\newblock Number 6032 in Lecture notes in computer science. Springer, Berlin,
  2010.

\bibitem{stefanovici_software-defined_2015} Ioan Stefanovici, Eno Thereska, Greg O'Shea, Bianca Schroeder, Hitesh Ballani,
  Thomas Karagiannis, Antony Rowstron, and Tom Talpey.
\newblock Software-defined caching: managing caches in multi-tenant data
  centers.
\newblock pages 174--181. ACM Press, 2015.

\bibitem{singh_server-storage_2008} Aameek Singh, Madhukar Korupolu, and Dushmanta Mohapatra.
\newblock Server-storage virtualization: integration and load balancing in data
  centers.
\newblock In {\em Proceedings of the 2008 {ACM}/{IEEE} conference on
  {Supercomputing}}, page~53. IEEE Press, 2008.

\bibitem{lall_data_2006} Ashwin Lall, Vyas Sekar, Mitsunori Ogihara, Jun Xu, and Hui Zhang.
\newblock Data streaming algorithms for estimating entropy of network traffic.
\newblock In {\em {ACM} {SIGMETRICS} {Performance} {Evaluation} {Review}},
  volume~34, pages 145--156. ACM, 2006.

\bibitem{meng_improving_2010} Xiaoqiao Meng, Vasileios Pappas, and Li~Zhang.
\newblock Improving the scalability of data center networks with traffic-aware
  virtual machine placement.
\newblock In {\em {INFOCOM}, 2010 {Proceedings} {IEEE}}, pages 1--9. IEEE,
  2010.

\bibitem{jun_ddos_2014} Jae-Hyun Jun, Cheol-Woong Ahn, and Sung-Ho Kim.
\newblock {DDoS} attack detection by using packet sampling and flow features.
\newblock pages 711--712. ACM Press, 2014.

\bibitem{yu_flowsense:_2013} Curtis Yu, Cristian Lumezanu, Yueping Zhang, Vishal Singh, Guofei Jiang, and
  Harsha~V. Madhyastha.
\newblock Flowsense: {Monitoring} network utilization with zero measurement
  cost.
\newblock In {\em International {Conference} on {Passive} and {Active}
  {Network} {Measurement}}, pages 31--41. Springer, 2013.

\bibitem{al-fares_hedera:_2010} Mohammad Al-Fares, Sivasankar Radhakrishnan, Barath Raghavan, Nelson Huang, and
  Amin Vahdat.
\newblock Hedera: {Dynamic} flow scheduling for data center networks.
\newblock In {\em Nsdi}, volume~10, pages 19--19, 2010.

\bibitem{mori_identifying_2004} Tatsuya Mori, Masato Uchida, Ryoichi Kawahara, Jianping Pan, and Shigeki Goto.
\newblock Identifying elephant flows through periodically sampled packets.
\newblock In {\em Proceedings of the 4th {ACM} {SIGCOMM} conference on
  {Internet} measurement}, pages 115--120. ACM, 2004.

\bibitem{ros-giralt_mathematical_2017} Jordi Ros-Giralt, Alan Commike, Sourav Maji, and Malathi Veeraraghavan.
\newblock A {Mathematical} {Framework} for the {Detection} of {Elephant}
  {Flows}.
\newblock {\em arXiv preprint arXiv:1701.01683}, 2017.

\bibitem{lakhina_characterization_2004} Anukool Lakhina, Mark Crovella, and Christiphe Diot.
\newblock Characterization of network-wide anomalies in traffic flows.
\newblock In {\em Proceedings of the 4th {ACM} {SIGCOMM} conference on
  {Internet} measurement}, pages 201--206. ACM, 2004.

\bibitem{lorenz_optimal_2003} Dean~H. Lorenz, Ariel Orda, and Danny Raz.
\newblock Optimal partition of {QoS} requirements for many-to-many connections.
\newblock In {\em {INFOCOM} 2003. {Twenty}-{Second} {Annual} {Joint}
  {Conference} of the {IEEE} {Computer} and {Communications}. {IEEE}
  {Societies}}, volume~3, pages 1670--1679. IEEE, 2003.

\bibitem{noauthor_mice_2013} networkheresy ~.
\newblock Of {Mice} and {Elephants}, November 2013.

\bibitem{pettit_open_2014} Justin Pettit.
\newblock Open {vSwitch} and the {Intelligent} {Edge}, 2014.

\bibitem{zhu_intelligent_2015} Huikang Zhu, Hongbo Fan, Xuan Luo, and Yaohui Jin.
\newblock Intelligent timeout master: {Dynamic} timeout for {SDN}-based data
  centers.
\newblock In {\em Integrated {Network} {Management} ({IM}), 2015 {IFIP}/{IEEE}
  {International} {Symposium} on}, pages 734--737. IEEE, 2015.

\bibitem{wang_expeditus:_2017} Peng Wang, Hong Xu, Zhixiong Niu, Dongsu Han, and Yongqiang Xiong.
\newblock Expeditus: {Congestion}-{Aware} {Load} {Balancing} in {Clos} {Data}
  {Center} {Networks}.
\newblock {\em IEEE/ACM Transactions on Networking}, 25(5):3175--3188, October
  2017.

\bibitem{wilson_better_2011} Christo Wilson, Hitesh Ballani, Thomas Karagiannis, and Ant Rowtron.
\newblock Better never than late: {Meeting} deadlines in datacenter networks.
\newblock {\em ACM SIGCOMM Computer Communication Review}, 41(4):50--61, 2011.

\bibitem{mittal_timely:_2015} Radhika Mittal, David Zats, Vinh~The Lam, Nandita Dukkipati, Emily Blem, Hassan
  Wassel, Monia Ghobadi, Amin Vahdat, Yaogong Wang, and David Wetherall.
\newblock {TIMELY}: {RTT}-based {Congestion} {Control} for the {Datacenter}.
\newblock pages 537--550. ACM Press, 2015.

\bibitem{lorenz_optimal_2002} Dean~H. Lorenz and Ariel Orda.
\newblock Optimal partition of {QoS} requirements on unicast paths and
  multicast trees.
\newblock {\em ieee/acm Transactions on Networking}, 10(1):102--114, 2002.

\bibitem{bari_data_2013} Md~Faizul Bari, Raouf Boutaba, Rafael Esteves, Lisandro~Zambenedetti Granville,
  Maxim Podlesny, Md~Golam Rabbani, Qi~Zhang, and Mohamed~Faten Zhani.
\newblock Data center network virtualization: {A} survey.
\newblock {\em IEEE Communications Surveys \& Tutorials}, 15(2):909--928, 2013.

\bibitem{roy_inside_2015} Arjun Roy, Hongyi Zeng, Jasmeet Bagga, George Porter, and Alex~C. Snoeren.
\newblock Inside the {Social} {Network}'s ({Datacenter}) {Network}.
\newblock pages 123--137. ACM Press, 2015.

\bibitem{benson_understanding_2009} Theophilus Benson, Ashok Anand, Aditya Akella, and Ming Zhang.
\newblock Understanding data center traffic characteristics.
\newblock page~65. ACM Press, 2009.

\bibitem{ahmed_novel_2008} Ejaz Ahmed, Andrew Clark, and George Mohay.
\newblock A {Novel} {Sliding} {Window} {Based} {Change} {Detection} {Algorithm}
  for {Asymmetric} {Traffic}.
\newblock pages 168--175. IEEE, October 2008.

\bibitem{box_time_2016} George E.~P. Box, Gwilym~M. Jenkins, Gregory~C. Reinsel, and Greta~M. Ljung.
\newblock {\em Time series analysis: forecasting and control}.
\newblock Wiley series in probability and statistics. John Wiley \& Sons, Inc,
  Hoboken, New Jersey, fifth edition edition, 2016.

\bibitem{kalekar_time_2004} Prajakta~S. Kalekar.
\newblock Time series forecasting using holt-winters exponential smoothing.
\newblock {\em Kanwal Rekhi School of Information Technology}, 4329008:1--13,
  2004.

\bibitem{beloglazov_optimal_2012} Anton Beloglazov and Rajkumar Buyya.
\newblock Optimal online deterministic algorithms and adaptive heuristics for
  energy and performance efficient dynamic consolidation of virtual machines in
  {Cloud} data centers: {ENERGY} {AND} {PERFORMANCE} {EFFICIENT} {DYNAMIC}
  {CONSOLIDATION} {OF} {VIRTUAL} {MACHINES}.
\newblock {\em Concurrency and Computation: Practice and Experience},
  24(13):1397--1420, September 2012.

\bibitem{schulz-zander_opensdwn:_2015} Julius Schulz-Zander, Carlos Mayer, Bogdan Ciobotaru, Stefan Schmid, and Anja
  Feldmann.
\newblock {OpenSDWN}: {Programmatic} {Control} over {Home} and {Enterprise}
  {WiFi}.
\newblock In {\em Proceedings of the 1st {ACM} {SIGCOMM} {Symposium} on
  {Software} {Defined} {Networking} {Research}}, {SOSR} '15, pages 16:1--16:12,
  New York, NY, USA, 2015. ACM.

\bibitem{clemm_data_2017} Alexander Clemm, Jan Medved, Robert Varga, Nitin Bahadur, Hariharan
  Ananthakrishnan, and Xufeng Liu.
\newblock A {Data} {Model} for {Network} {Topologies}.
\newblock Internet-{Draft} draft-ietf-i2rs-yang-network-topo-20, Internet
  Engineering Task Force, December 2017.

\bibitem{jurgen_schonwalder_network_2012} {J{\"u}rgen Sch{\"o}nw{\"a}lder}.
\newblock Network {Configuration} {Management} with {NETCONF} and {YANG}, July
  2012.

\bibitem{bjorklund_yang_2010} Martin Bjorklund.
\newblock {\em {YANG} - {A} {Data} {Modeling} {Language} for the {Network}
  {Configuration} {Protocol} ({NETCONF})}.
\newblock Number 6020 in Request for {Comments}. RFC Editor, October 2010.
\newblock Published: RFC 6020.

\bibitem{enns_network_2011} Rob Enns, Martin Bjorklund, Andy Bierman, and J{\"u}rgen Sch{\"o}nw{\"a}lder.
\newblock {\em Network {Configuration} {Protocol} ({NETCONF})}.
\newblock Number 6241 in Request for {Comments}. RFC Editor, June 2011.
\newblock Published: RFC 6241.

\bibitem{maxim_novak_serialization_2014} {Maxim Novak}.
\newblock Serialization {Performance} comparison ({C}\#/.{NET}) {\textendash}
  {Formats} \& {Frameworks} ({XML}{\textendash}{DataContractSerializer} \&
  {XmlSerializer}, {BinaryFormatter}, {JSON}{\textendash} {Newtonsoft} \&
  {ServiceStack}.{Text}, {Protobuf}, {MsgPack}), March 2014.

\bibitem{bruno_krebs_beating_nodate} {Bruno Krebs}.
\newblock Beating {JSON} performance with {Protobuf}.

\bibitem{ahmad_arima_2013} Wan Kamarul Ariffin~Wan Ahmad and Sabri Ahmad.
\newblock Arima model and exponential smoothing method: {A} comparison.
\newblock {\em AIP Conference Proceedings}, 1522(1):1312--1321, 2013.

\bibitem{hyndman_another_2006} Rob~J. Hyndman and Anne~B. Koehler.
\newblock Another look at measures of forecast accuracy.
\newblock {\em International Journal of Forecasting}, 22(4):679 -- 688, 2006.

\bibitem{choudhary_runtime-efficacy_2017} Dhruv Choudhary, Arun Kejariwal, and Francois Orsini.
\newblock On the {Runtime}-{Efficacy} {Trade}-off of {Anomaly} {Detection}
  {Techniques} for {Real}-{Time} {Streaming} {Data}.
\newblock {\em arXiv preprint arXiv:1710.04735}, 2017.


\end{thebibliography}

%%%%% CLEAR DOUBLE PAGE!
\newpage{\pagestyle{empty}\cleardoublepage}

\end{document}
